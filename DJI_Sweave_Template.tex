\documentclass[12pt]{article}         % the type of document and font size (default 10pt)
\usepackage[margin=1.0in]{geometry}   % sets all margins to 1in, can be changed
\usepackage{moreverb}                 % for verbatimtabinput -- LaTeX environment
\usepackage{url}                      % for \url{} command
\usepackage{amssymb}                  % for many mathematical symbols
\usepackage[pdftex]{lscape}           % for landscaped tables
\usepackage{longtable}                % for tables that break over multiple pages
\title{Demographic and Economic Leading Indicators}  % to specify title
\author{Bob McPherson}          % to specify author(s)
\usepackage{Sweave}
\begin{document}                      % document begins here
\Sconcordance{concordance:DJI_Sweave_Template.tex:DJI_Sweave_Template.Rnw:%
1 10 1 1 0 31 1 1 214 22 1 1 198 22 1 1 11 5 0 1 3 13 1}


% If .nw file contains graphs: To specify that EPS/PDF graph files are to be 
% saved to 'graphics' sub-folder
%     NOTE: 'graphics' sub-folder must exist prior to Sweave step
%\SweaveOpts{prefix.string=graphics/plot}

% If .nw file contains graphs: to modify (shrink/enlarge} size of graphics 
% file inserted
%         NOTE: can be specified/modified before any graph chunk
\setkeys{Gin}{width=1.0\textwidth}

\maketitle              % makes the title
%\tableofcontents        % inserts TOC (section, sub-section, etc numbers and titles)
%\listoftables           % inserts LOT (numbers and captions)
%\listoffigures          % inserts LOF (numbers and captions)
%                        %     NOTE: graph chunk must be wrapped with \begin{figure}, 
%                        %  \end{figure}, and \caption{}
%%%%%%%%%%%%%%%%%%%%%%%%%%%%%%%%%%%%%%%%%%%%%%%%%%%%%%%%%%%%%%%%%%%%
% Where everything else goes

\section{How to typeset \textsf{R} code}

If you want to see both the input and output, do this:

\begin{Schunk}
\begin{Sinput}
> runif(10)
\end{Sinput}
\begin{Soutput}
 [1] 0.87640976 0.75944848 0.68631318 0.47581328 0.04821688 0.08480474
 [7] 0.02774285 0.76877110 0.19790805 0.57814911
\end{Soutput}
\end{Schunk}

If you want to see output, but no input, do this:

\begin{Schunk}
\begin{Soutput}
 [1] 0.2817267787 0.1075962023 0.1494490174 0.4508275751 0.3554472025
 [6] 0.0009781977 0.6655434514 0.9622834574 0.7886318574 0.6234380030
\end{Soutput}
\end{Schunk}

If you want to see input, but no output, do this:

\begin{Schunk}
\begin{Sinput}
> runif(13)
\end{Sinput}
\end{Schunk}

If you want to run some \textsf{R} code but hide the input/output from the reader then you can do both at the same time:


\bigskip   % leave some empty space (optional)

and you can double-check that it worked later (if you like)

\begin{Schunk}
\begin{Sinput}
> x  # use keep.source=TRUE if you want comments printed
\end{Sinput}
\begin{Soutput}
 [1]  2  3  4  5  6  7  8  9 10 11
\end{Soutput}
\begin{Sinput}
> y
\end{Sinput}
\begin{Soutput}
 [1] 0.21601262 0.08723895 0.97032938 0.24798698 0.89185310 0.17696276
 [7] 0.09052184 0.79005603 0.89647824 0.54800134
\end{Soutput}
\end{Schunk}

If you want to write some \textsf{R} code but not have it evaluated at all then do this:

\begin{Schunk}
\begin{Sinput}
> # whatever you write here must be syntactically correct R code
> runif(1000000000000000000000000)
\end{Sinput}
\end{Schunk}

If you would like to include a figure that's generated completely by \textsf{R} code, then you can do something like the following.

\begin{figure}
\includegraphics{DJI_Sweave_Template-007}
\caption{Here is the plot we made}
\end{figure}


Sometimes we would like the output to look like \LaTeX\ output instead of \textsf{R} output.  In that case, do the following.

\begin{Schunk}
\begin{Sinput}
> library(xtable)
> xtable(summary(lm(y ~ x)), caption = "Here is the table we made")
\end{Sinput}
% latex table generated in R 3.5.1 by xtable 1.8-2 package
% Tue Aug 21 00:19:27 2018
\begin{table}[ht]
\centering
\begin{tabular}{rrrrr}
  \hline
 & Estimate & Std. Error & t value & Pr($>$$|$t$|$) \\ 
  \hline
(Intercept) & 0.2330 & 0.2866 & 0.81 & 0.4399 \\ 
  x & 0.0398 & 0.0403 & 0.99 & 0.3529 \\ 
   \hline
\end{tabular}
\caption{Here is the table we made} 
\end{table}\end{Schunk}


\end{document}
